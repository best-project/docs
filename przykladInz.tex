\documentclass[twoside]{projektInzynierskiMS}
\usepackage[polish]{babel}
\usepackage[cp1250]{inputenc}
\usepackage{amsmath}
%\drukJednostronny

%% tytu? promotor iautor (\title to komenda standardowa)
\title{Aplikacja Edukacyjna}
\promotor{dr Mariusz Pleszczynski}


%% ka?dy autor musi mie? 4 argumenty: imi? nazwisko, nr albumu, procent wk?adu, opis wk?adu
\autor{Adam Wojkowski}{112233}{12} {Wk?ad pracy tego autora}
	
\autor{Malwina Borecka-Xsinska}{112233}{30}
	{Wk?ad pracy tego autora dalej,
	Wk?ad pracy tego autora dalej,
	Wk?ad pracy tego autora dalej,
	
	Wk?ad pracy tego autora dalej,
	Wk?ad pracy tego autora dalej,
	Wk?ad pracy tego autora dalej,
	Wk?ad pracy tego autora dalej,
	Wk?ad pracy tego autora dalej,
	Wk?ad pracy tego autora dalej,
	Wk?ad pracy tego autora dalej,
	Wk?ad pracy tego autora dalej,
	}

\autor{Stanis?aw Morski}{112233}{40}	
{Wk?ad pr. Morskiego}
	
	


%% dedykacja mile widziana
\dedykacja{To jest\\dedykacja}
%\NumeryNaPoczatku
%% numeracja wzor?w tu w??czona typu (1.2.3), ta druga to typu (1.2), domy?lnie typu (1)
%\subsectionWzory
% \sectionWzory  

%\rozdzialy


%\literowaNumeracjaDodatkow %% w??czy numeracj? dodatk?w literami
%\rzymskaNumeracjaDodatkow  %%w??czy numeracj? dodatk?w liczbami rzymskimi

%% wy??czenie wyja?nie?:
\bezWyjasnien

%% standardowe komendy \newtheorem  dzia?aj? jak woryginale
\newtheorem{tw}{Twierdzenie}%[subsection]
\newtheorem{twa}{Twierdzenie}%[section]
\newtheorem{dd}{Definicja}%[subsection]

\begin{document}




wst?p jest tu 



\section{Tytu? rozdzia? pierwszego}


Tu jest wn?trze rozdzia?u.

\begin{twa}
Twierdzenie Twierdzenie Twierdzenie Twierdzenie Twierdzenie 
\end{twa}
\begin{dd}
Twierdzenie Twierdzenie Twierdzenie Twierdzenie Twierdzenie 
\end{dd}

\thesection,\thesubsection,\thesubsubsection,

\subsection{Tytu? podrozdzia?u}
Francuz?w sto woz?w sieci purpurowe kwiaty ka?dy mimowolnie porz?dku pilnowa?
\begin{equation}
a^2+b^2=c^2.
\end{equation}

\begin{tw}
Twierdzenie Twierdzenie Twierdzenie Twierdzenie Twierdzenie 
\end{tw}

\begin{dd}
Twierdzenie Twierdzenie Twierdzenie Twierdzenie Twierdzenie 
\end{dd}
\begin{equation}
E=mc^2.
\end{equation}

Lorem Ipsum generator (\texttt{http://lipsum.pl/index.php});
Gdyby ?y? d?u?ej, mo?e nas wytuza. U nas towarzystwo liczne od m?czyzn dziwi! C? z?ego, ?e tamuje progresy, ?e spud?uje. szarak, gracz nie wa??. Wi?c by?o rz?d ruszy? lub bez ogona jest pewna odmiana. Trzeba si? strzelbami po ?acinie. M?czyznom dano jako w~domu podobnych spraw nie ma obszerno?? dostatecznej dla s?ug zapyta?. Odemkn??, wbieg? do dworu. Tu ?miech m?odzie?y mow? Wojskiego Wo?ny trybuna?u. Takie by?y r?czki, co jasnej bronisz Cz?stochowy staj? mu przed Kusym o rze?biarstwie! Dowiod?a, ?e by?y zabawy, spory o politycznych sprawach rozmawia? po k?dzieli a~oni tak nas wytuza. U nas towarzystwo liczne od kilku dzieje tego dnia powiada?. Dobrze, m?j s?siedzie maj?tek bratni wszystko si? trzeba, pos?pny obok srebrnych, od Nil sz?a hucz?c ku drzwiom odprowadza? kt?rej rami? z Wizgirdem dominikanie z wiecz?r gospodarz widzi, polskiej szacie siedzi jak ?aczek przed nauczycielem. Szcz?ciem, ?e go my?lano do zamku sie? wielka, jeszcze gorzej! Teraz wszed? do nowej s?siadki brano z obcego klasztor przyszed?, szanowne damy. Pan ?wiata wie, ?e mia? g?os zabiera?. Umilkli wszyscy o nich.

Francuz?w sto woz?w sieci purpurowe kwiaty ka?dy mimowolnie porz?dku pilnowa?. Bo nie ma jutro sam lat dziesi?? by?em wtenczas wszyscy s?uchali tabakierk? z?ot? na piasku, bez grzeczno?ci rozprawiali, nieco wylot?w kontusza nala? w?grzyna z drzewa, lecz na drobnych ?ladach zatrzymywa? my?la? o palec. Wiedzia?em, ?e pewnie mia?a wysmuk??, kszta?tn?, pier? powabn? sukni? materyjaln?, r?ow?, jedwabn? gors wyci?ty, ko?nierzyk z nim odszed?, wyskoczy? na parkanie sta?a wypisana niesrogi. Odgadn?a s?siadka za?mia?a si? damom, starcom cofn?? si?, jak pieni?dze ?ydzi. To nie wida? n?ki na drobnych ?ladach zatrzymywa? my?la? o tym obrazem. W?a?nie rzecz swoj? tokowa bez?adnie. nieporz?dek mi?y! Niestare by?y zaj?te sto?u przywo?awszy dwie twarze paryskich kawiarniach. Bo nie rzuca Petersburgu mieszka?a przed laty. Wchodzi, cofn?? si?, wieczerz? przy Bernardynie, bernardyn zm?wi? kr?tki pacierz po duszy, cz?sto bez nogi, przyj?wszy ja?mu?n? stan?? cyfr? powi?zany p?otek po?yska? si? zabawia? lubi? gesta). Teraz nie jest jak O?tarzyk z?oty zawsze Zabo bieg?y przed laty, nad uchem. Tadeusz przygl?da? si? tajemnie, ?cigany od chmielu tyki tem miejscu swem siad? przy niej.

Suwar?w w~domu wiecznie b?dzie z opieki panicz bogaty, krewny pa?ski lekka jak mnie to m?wi?c, ?e Litwie Wo?ny pas s?ucki, pas lity przy zachodzie wszystko strwoni?, na ostrym ko?cu dzieje domowe powiatu dawano przez Niemen rzek?: Wielmo?ni Szlachta, Bracia Dobrodzieje! Forum my?liwskiem tylko a? do dworu uprawne dobrze zachowana sklepienie ca?e weso?o, lecz podmurowany. ?wieci?y si? przyci?gn?? do domu, fortuny szczodrot obja?niaj? wrodzone wdzi?ki nigdy nie rzuca Litwie chodzi? po k?dzieli potem si? pan Hrabia ma jutro sam kr?l j? j?zyku. Tak ka?e przyzwoito??). nikt tam zamek dzi? toczy si? ?eni? ju? robi? projekt, ?e zna r?wnie p?dzel, noty, druki. A? os?upia? Tadeusz przygl?da? si? strzelbami pani ta t?uszcza. Bo nie poruczy, bo tak wedle dzisiejszej mody je?dzi? na stosach Moskali siek?c wrog?w, mi?dzy szlacht? dzieje chciano zamkn?? ca?ej ozdobi widz? wst?g jasnych p?ki. Ta przerwa rozm?w trwa?a ju? robi? projekt, ?e tytu?y przychodz? z wolna krocz stado cielic tyrolskich z Bonapart?. tu Ryk?w przerwa? sukienka bia?a, ?wie?o z ko?ka zdj?ty do Podkomorzanki. Nie zmienia czy.

Moskalom przez konar b?ysn?o jako po zadzwonieniu na wzmiank? Warszawy rzek?, podnios?szy g?ow?: Pan ?wiata wie, ?e gotyckiej s? nasze spraw bernardy?skie. c? by znaczy? po?czochach, ze ?wiecami uczciwo?ci, oknie sta? patrz?c, dumaj?c wonnymi powiewami kwiat?w oddychaj?c oblicze a? do dworu. Tu ?miech m?odzie?y mow? Wojskiego zag?uszy?. Wstano od dzisiaj nie dozwala?, by tu Ryk?w przerwa? s?udzy. ogl?da czule, jako ?wieca przez grzeczno?? nie po duszy, ja wam s?u?y?, moje panny c?rki cho? suknia kr?tka, oko pa?skie konia tuczy. Wojski towarzystwa nam si? damom, starcom tu? nad b??kitnym Niemnem rozci?gnionych. Do zobaczenia! tak gada?: C? z?ego, ?e tamuje progresy, ?e mi wybaczy, ?e po?czochach, ze srebrnymi klamrami trzewiki peruka z nowych go?ci. takim Litwinka tylko si? jak mnich na prawo, kozio?ka, z brabanckich koronek poprawia?a, to m?wi?c, ?e odg?os tr?bki ubiory. By?a to m?wi?c, ?e jacy? Francuzi wymowny zrobili wynalazek: i? ludzie s? architektury. Cho? S?dzia tu? to m?wi?c, ?e przeszkadza kulturze, ?e jacy? Francuzi wymowny zrobili wynalazek: i? ludzie s? architektury. Cho? S?dzia wie, ?e nauczyciel ?adny dwiestu.

%% UWaga na \newlineTekst oraz \newlineSpis. Mo?na te? u?y? \newline, dzia?a jak \newlineSpis\newlineTekst
\section[Tytu? drugiego rozdzia?u. Bardzo d?ugi \ldots]
        {Tytu? drugiego rozdzia?u. \newlineTekst Bardzo d?ugi tytu?. \newlineTekst
          Jego \newlineSpis formatowanie jest trudniejsze}

Tu jest wn?trze rozdzia?u drugiego.          


Przeprosiwszy go powita?. Dawno domu dostatek mieszka ?e zamkowej sieni siad? przy niej z dala, r?ce przy boku s?siadki reszt? rozdzielono mi?dzy rz?dem sta?y spisane sprawy, kt?re wylotem kontusz otar? pr?dko, jak d?umy jakiej ca?y las d?ugie zwija?y si? ko?em. takim Litwinka tylko si? ta chwa?a nale?y chartu Soko?owi. Pytano zdania bo tak szanownych go?ci. zamku sie? wielka, jeszcze dobrze na miejscu pustym oczy zmru?y? pu?ku gadano, jak wytnie dwa tysi?ce jako ?wieca przez p?otki, przez to m?wi?c, ?e nas towarzystwo ca?e zniszczone sekwestrami rz?du bez?adno?ci? opieki, wyrokami s?du kt?rym ogie? p?on??. R?wnie? patrzy?a ona, z harbajtelem zawi?zanym ko?cu z kt?rych by przy kt?rym wszystko oddycha?o. Kr?tkie by?y zaj?te sto?u przywo?awszy dwie strony: Uciszcie si?! wo?a. Marz?c knieje wi?c ja wkr?tce sprawi? ci wesele. Jest s?awa, on Pana zast?puje cho?odziec litewski milcz?c ?wawo jedli. , cho? m?odzie? nieraz na urz?d wielkie polowanie. te? szlachecka. Grzeczno?? nie poruczy, bo tak krzycz?c pan Podkomorzy hec! od rana wiedzia?, czy go wkr?tce sprawi? ci znowu.

Ojczy?nie Boga, przodk?w wiar? prawa zabawia? lubi? por?wnywa?, a~oni tak szanownych go?ci. biegu dotkn?a blisko naszego m?odziana. Uczepiwszy falban? o ?yciu, o ?mierci syna. Bra? dom nikt nigdy na Francuza, ?e mia? by? siedzeniem. Rumieni? si?, wieczerz? przy stole. To jedno puste miejsce, jak od Moskali, skaka? kry? si? zabawia? go?ci ?ydom do sto?u przywo?awszy dwie strony: Uciszcie si?! wo?a. Marz?c z Rymsz?, Rymsza z tych imion spisem wo?nemu jest ni? si? zabawia? go?ci nie by? ruchawy od m?czyzn zalety ?ci?gn?y wzrok surowy zna? cz?owieka nie bieg? s?ug zapyta?. Odemkn??, wbieg? do domu, fortuny szczodrot obja?niaj? wrodzone wdzi?ki m?oda. Jej zjawienie si? rumieniec oblek?y. Tadeusz, by nie je?dzi? na spoczynek powraca. Ju? S?dzia Podkomorzego zda? si? wachlowa?a, to m?wi?c, ?e oko nie mo?e. Wida?, ?e zamczysko wzi?li?my pukle nie powiedzia? cho? ?wiadka nie dozwala?, by wychowanie niczego nie zdradzi? swego roztargnienia: Prawda - kanonada. Ruskie przys?owie: z powozu. konie porzucone same widzi sprz?ty, te? same szczypi?c traw? ci?gn?y powoli pod bram?. We dworze jako naprzykrzona mucha. Pragn??by u wieczerzy? S? tu pan S?dzia.

M?wi?c, Podkomorzemu ?cisn?? za rarogiem zazdroszczono domowi, przed nim le?y Fedon Suwar?w naukach mniej krzykliwy wion?a ogrodem przez Niemen rzek?: Dzi?, nowym zwyczajem my na konikach ma?e ?arciki umia? komponowa? i?by je wicher rozerwie dziwniejsze od kilku dzieje chciano przeczy? chwa?y. Wi?c do z?otego run on Pana M?wi?c, Podkomorzemu ?cisn?? za rarogiem zazdroszczono domowi, przed nim dla zabawki Bo nie rozwity, lecz latem nic nie jest obora. Dozoru tego dnia powiada?. Dobrze, m?j Tadeuszu, ?e? si? Soplica. wszyscy siedli jelenie rogi z Rymsz?, Rymsza z pola. Tak ka?e u Niemna odebra? wiadomo??. mo?e te? same portrety na filarach, pod?oga wys?ana kamieniem, ?ciany bez r?ki lub papug? Pa?skim pisano zakonie opisuj?, bo tak nazywano m?odzie?ca, kt?ry teraz za nim spostrzeg? si?, serce mu jak wi?nie bli?ni?ta. U tej znalaz? podobne oczy, usta, lica. czasie wojny otoczony chmur? pu?k?w, tysi?cem dzia? zbrojny wprz?g?szy ko?cu ?r?d biesiadnik?w siedzia? s?ucha? zmru?ywszy oczy, usta, lica. sieni siad? przy jego wiernym ludem! Jak go pilnowa? oczy wko?o obraca? ostr?ne. Gdy si? chlubi zaj?c jak.

\dodatek{M?j specjalny dodatek}

Tu tre?? dodatku. Zwr??my uwag? na spos?b numerowania dodatku, 
mo?liwa jest zmiana numerowania, patrz wyja?nienia.
          
%% to wpisuje si? do spisu tre?ci, ale bez numeru rozdzia?u,
%% mo?na te? u?ywa? \dodatek{Tytu?}, kt?ry jest numerowany, ale inaczej ni? rozdzia?y.
\dodatkowo{Rysunki}

Tu rysunki

\dodatkowo{Programy}

Tu programy

\begin{verbatim}
#include <stdio.h>

int main()
{
   printf("Hello world\n");
}
\end{verbatim}

\noindent
Oraz 

\bigskip

\vrule\hspace{10pt}\begin{minipage}{10cm}
\begin{verbatim*}
<?php
   echo "test=$test";
?>
\end{verbatim*}
\end{minipage}

\begin{tw}
Twierdzenie Twierdzenie Twierdzenie Twierdzenie Twierdzenie 
\end{tw}
\begin{thebibliography}{12}

\bibitem{PozNazwa1} Jaka? pozycja literatury
\bibitem{InnPoz} Jaka? pozycja literatury

\end{thebibliography}
\end{document}
